% \documentclass[9pt,a4paper]{jsarticle}                % for upLaTeX
\documentclass[9pt,a4paper]{ltjsarticle}                % for LuaTeX
% \documentclass[9.5pt,a4paper]{ltjsarticle}            % for 小数指定はできなかった

% ###############################
% 航空宇宙学会年会 原稿
% ###############################

\pagestyle{empty}       % ページ番号を消す

\usepackage[top=25truemm,bottom=25truemm,left=23truemm,right=23truemm]{geometry}

% フォント設定
% https://ja.osdn.net/projects/luatex-ja/wiki/LuaTeX-ja%E3%81%AE%E4%BD%BF%E3%81%84%E6%96%B9
% http://0-chromosome.hatenablog.jp/entry/2015/08/10/214938
\usepackage[no-math]{fontspec}
% ↑これつけるとsiunitxのmicroが消えた... ↓下で解決
\usepackage[yu-win,deluxe]{luatexja-preset}

% https://ja.osdn.net/projects/luatex-ja/wiki/LuaTeX-ja%E3%81%AE%E4%BD%BF%E3%81%84%E6%96%B9
% https://tex.stackexchange.com/questions/25249/how-do-i-use-a-particular-font-for-a-small-section-of-text-in-my-document
% http://arxiv.hatenablog.com/entry/2016/11/30/183000
\usepackage{luatexja-fontspec}
% \newfontfamily\fontConsolas{Consolas}
% \newfontfamily\myFontYuGothM[ExternalLocation,Scale=0.91]{YuGothM.ttc}  % これでもできた
% \newjfontfamily\myJFontYuGothM[ExternalLocation]{YuGothM.ttc}           % これでもできた
\newfontfamily\myFontYuGothM[ExternalLocation,Scale=0.91]{YuGothM001.ttf}
\newjfontfamily\myJFontYuGothM[ExternalLocation]{YuGothM001.ttf}
\newfontfamily\myFontYuGothB[ExternalLocation,Scale=0.91]{YuGothB001.ttf}
\newjfontfamily\myJFontYuGothB[ExternalLocation]{YuGothB001.ttf}
% \newfontfamily\myFontYuGothM[ExternalLocation,Scale=MatchLowercase]{YuGothM.ttc}   % サイズが上手く合わない
% \newjfontfamily\myJFontYuGothM[ExternalLocation,Scale=MatchLowercase]{YuGothM.ttc} % サイズが上手く合わない

\usepackage{amsmath,amssymb}
\usepackage{siunitx}                            % SI単位
\sisetup{input-ignore={,},input-decimal-markers={.},group-separator={,},group-minimum-digits=4,group-digits=integer}
% \sisetup{mathsmu = \text{μ},textmu  = μ}    % μとµは違う
\sisetup{math-micro=\text{µ},text-micro=µ}
% μ
% U+03BC
% µ
% U+00B5
% for % ↑これつけるとsiunitxのmicroが消えた... ってのの解決
\usepackage{bm}                                 % for vector    % font系パッケージの後に宣言
\usepackage{setspace}                           % for spacing環境
\usepackage[normalem]{ulem}                     % for \sout{} 取り消し線
% ulemの下線注意
% http://kawaiihaseigi.blogspot.jp/2013/01/ulemstyemph.html
\usepackage{url}
% http://osksn2.hep.sci.osaka-u.ac.jp/~naga/miscellaneous/tex/tex-tips3.html
\usepackage{booktabs}                           % for \toprule \midrule \bottomrule
% \usepackage[dvipdfmx]{graphicx}               % for upLaTeX
\usepackage[luatex]{graphicx}                   % for LuaTeX
\usepackage{multirow}                           % 表で行結合
% https://texwiki.texjp.org/?hyperref
% http://www.biwako.shiga-u.ac.jp/sensei/kumazawa/tex/hyperref.html
% http://0-chromosome.hatenablog.jp/entry/2015/04/10/175912
% LuaTeX. upLaTeXではオプションが違う.
\usepackage[unicode=true,hidelinks,bookmarksnumbered=true,bookmarksdepth=subsubsection,bookmarksopen=true]{hyperref}
\usepackage{comment}
% 画像強制位置合わせ
% http://ac206223.ppp.asahi-net.or.jp/adiary/memo/adiary.cgi/hirosugu/TeX%E3%81%A7%E5%9B%B3%E3%82%92%E3%82%B3%E3%83%BC%E3%83%89%E8%A8%98%E8%BF%B0%E4%BD%8D%E7%BD%AE%E3%81%AB%E5%BC%B7%E5%88%B6%E7%9A%84%E3%81%AB%E5%87%BA%E5%8A%9B%E3%81%99%E3%82%8B
\usepackage{here}

% sectionのフォントサイズ変更
% https://qiita.com/roaming_south/items/ea39b9554167fcaba882
\usepackage{titlesec}
\titleformat*{\section}{\normalsize\bfseries}
\titleformat*{\subsection}{\normalsize\rmfamily}
\titleformat*{\subsubsection}{\normalsize\rmfamily}

% http://www.biwako.shiga-u.ac.jp/sensei/kumazawa/tex/mathbfbb.html
% mathpzc
\DeclareMathAlphabet{\mathpzc}{OT1}{pzc}{m}{it}

% 引用[]を右上に 方法3
% https://tex.stackexchange.com/questions/94178/temporarily-disable-superscript-in-citation
\usepackage[square]{natbib}

% 2段組み
% http://nezumitori.hatenablog.jp/entry/20160509/1462762800
\usepackage{multicol}

% 2段組み図 方法2
% https://tex.stackexchange.com/questions/12262/multicol-and-figures
% \usepackage[labelsep=none]{caption}
\usepackage[labelsep=quad]{caption}
\newenvironment{Figure}
  {\par\medskip\noindent\minipage{\linewidth}}
  {\endminipage\par\medskip}
\newenvironment{Table}
  {\par\medskip\noindent\minipage{\linewidth}}
  {\endminipage\par\medskip}
% キャプションのエラー
% Package caption Warning: The option `hypcap=true' will be ignored for this(caption)
% については
% https://tex.stackexchange.com/questions/32344/issue-with-capt-of-package

% コンパイル時に図をとばす
% http://blog.livedoor.jp/tmako123-programming/archives/42210596.html
\newif\iffigure
\figurefalse
\figuretrue


% http://d.hatena.ne.jp/gp98/20090919/1253367749
\makeatletter % プリアンブルで定義開始

% 表示文字列を"図"から"Figure"へ
\renewcommand{\figurename}{Fig.\,}
\renewcommand{\tablename}{Table\,}

% 図番号を"<節番号>.<図番号>" へ
\renewcommand{\thefigure}{\thesection.\arabic{figure}}
\renewcommand{\thetable}{\thesection.\arabic{table}}
\renewcommand{\theequation}{\thesection.\arabic{equation}}

% 節が進むごとに図番号をリセットする
\@addtoreset{figure}{section}
\@addtoreset{table}{section}
\@addtoreset{equation}{section}

\makeatother % プリアンブルで定義終了


% 引用[]を右上に 方法3
% https://tex.stackexchange.com/questions/94178/temporarily-disable-superscript-in-citation
\setcitestyle{super}
\setcitestyle{citesep={,}}
\DeclareRobustCommand*{\citen}[1]{%
    \begingroup
        \romannumeral-`\x % remove space at the beginning of \setcitestyle
        \setcitestyle{numbers}%
        \cite{#1}%
    \endgroup
}

% 数式記号命令 \mhyph の定義
% ハイフン
% http://d.hatena.ne.jp/zrbabbler/20160609/1465426138
\DeclareMathSymbol{\mhyph}{\mathalpha}{operators}{`-}


% マクロ
\newcommand{\refSec}[1]   {\ref{#1}章}
\newcommand{\refApd}[1]   {付録\ref{#1}}
\newcommand{\refSsec}[1]  {\ref{#1}節}
\newcommand{\refSssec}[1] {\ref{#1}項}
\newcommand{\refFig}[1]   {\figurename\ref{#1}}
\newcommand{\refTable}[1] {\tablename\ref{#1}}
\newcommand{\refEq}[1]    {Eq.\,\ref{#1}}

% http://www.latex-cmd.com/equation/max_min.html
\newcommand{\argmax}{\mathop{\rm arg~max}\limits}
\newcommand{\argmin}{\mathop{\rm arg~min}\limits}

% \midの拡張
% http://d.hatena.ne.jp/zrbabbler/20120411/1334151482
% \newcommand{\relmiddle}[1]{\mathrel{}\middle#1\mathrel{}}
\newcommand{\relmiddle}[1]{\mathrel{}\middle#1}

% https://en.wikibooks.org/wiki/LaTeX/Mathematics#Fractions_and_Binomials
\newcommand*\rfrac[2]{{}^{#1}\!/_{#2}}

\newcommand{\mySection}[1] {%
    \section{#1}%
    \vspace{-6pt}
}
\newcommand{\mySubsection}[1] {%
    \subsection{#1}%
    \vspace{-6pt}
}
\newcommand{\mySubsubsection}[1] {%
    \subsubsection{#1}%
}
\newcommand{\mySubsubsubsection}[1] {
    \vspace{5pt}
    \noindent
    ・#1 \par
}

% 表見出しフォント
\newcommand{\myJFontTable}[1] {\myJFontYuGothM{#1}}
\newcommand{\myEFontTable}[1] {\myFontYuGothM{#1}}
\newcommand{\myFontTable}[1] {\myJFontYuGothM{\myFontYuGothM{#1}}}

\allowdisplaybreaks[4]

\usepackage{lipsum}
\usepackage {bxjalipsum}

\begin{document}

\begin{center}
    {\fontsize{16pt}{24pt}\selectfont \textbf{1A23 ほげほげほげほげによるぴよぴよぴよぴよのための}} \\[8pt]
    {\fontsize{16pt}{24pt}\selectfont \textbf{ふがふがふがふがに関する研究}} \\[14pt]

    ○\textbf{田中 太郎}(東京大学),\textbf{鈴木 一郎}(京都大学), \\
    \textbf{高橋 修},\textbf{松本 次郎},\textbf{佐々木 明},\textbf{内藤 康平}(東京大学) \\[14pt]

    Hogehoge Piyopiyo Fugafuga of Hogepiyofuga Study \\

    Taro Tanaka (The University of Tokyo), Ichiro Suzuki (Kyoto University), \\
    Osamu Takahashi, Jiro Matsumoto, Akira Sasaki, Kohe Naito (The University of Tokyo) \\[14pt]

    Key words :  Aircraft, Safety and Reliability, Remote Sensing, \\
    Flight Dynamics, Artificial Satellites \\[14pt]

    \textbf{Abstract}
\end{center}
\vspace{-2mm}

\lipsum[1-2]

\begin{multicols}{2}

\mySection{序論 \label{Sec:Intro}}
\mySubsection{研究背景}
\begin{Figure}
    \centering
    \iffigure
    \rule{50mm}{30mm}
    % \includegraphics[width=50mm]{./img/hoge.pdf}
    \fi
    \captionsetup{type=figure}
    \caption{``Hoge'' イメージ図}
\end{Figure}
\jalipsum[1-2]{wagahai}

\mySubsection{サブセクション}
\mySubsubsection{サブサブセクション}
\jalipsum[3]{wagahai}


\mySubsubsection{サブサブセクション}
\mySubsubsubsection{サブサブサブセクション}
ああああ.


\mySubsubsubsection{サブサブサブセクション}
いいいい.
% \vfill\null
% \columnbreak


\mySection{図表}
\mySubsection{図}
\begin{Figure}
    \centering
    \iffigure
    \rule{50mm}{30mm}
    % \includegraphics[width=50mm]{./img/hoge.pdf}
    \fi
    \captionsetup{type=figure}
    \caption{大きな図}
\end{Figure}
\begin{Figure}
    \begin{minipage}{0.49\hsize}
        \centering
        \iffigure
        \rule{35mm}{30mm}
        % \includegraphics[width=30mm]{./img/hoge.pdf}
        \fi
    \end{minipage}
    \begin{minipage}{0.49\hsize}
        \centering
        \iffigure
        \rule{35mm}{30mm}
        % \includegraphics[width=30mm]{./img/hoge.pdf}
        \fi
    \end{minipage}
    \captionsetup{type=figure}
    \caption{小さな図(左)と小さな図(右)}
\end{Figure}
\begin{Figure}
    \begin{minipage}{0.49\hsize}
        \centering
        \iffigure
        \rule{35mm}{30mm}
        % \includegraphics[width=30mm]{./img/hoge.pdf}
        \fi
    \end{minipage}
    \begin{minipage}{0.49\hsize}
        \centering
        \iffigure
        \rule{35mm}{30mm}
        % \includegraphics[width=30mm]{./img/hoge.pdf}
        \fi
    \end{minipage}
    \captionsetup{type=figure}
    \caption{小さな図(左)と小さな図(右)}
\end{Figure}


\mySubsection{表}
\begin{Table}
    \centering
    \renewcommand{\arraystretch}{0.8}
    \captionsetup{type=table}
    \caption{表テンプレート}
    \small
    \begin{tabular}{lccc}
    \toprule
    \multicolumn{1}{c}{\myJFontTable{}}       &
    \multicolumn{1}{c}{\myJFontTable{記号}}   &
    \multicolumn{1}{c}{\myJFontTable{単位}}   &
    \multicolumn{1}{c}{\myJFontTable{値}} \\
    \midrule
    \myJFontTable{ほげ直径} &
        $D$         & $\si{m}$           & $1.234$ \\
    \myJFontTable{ピヨ数} &
        $Q$         &                    & $6$ \\
    \myFontTable{Fuga値} &
        $f$         & $\si{m}$           & $3.56$ \\
    \myFontTable{HOGE精度} &
        $\delta_h$  & $\si{\micro m}$    & $1.00$ \\
    \myFontTable{Piyo精度} &
        $\delta_f$  & $\si{\micro rad}$  & $2.00$ \\
    \bottomrule
    \end{tabular}
    \renewcommand{\arraystretch}{1}
\end{Table}


\mySection{出典}
\citen{Schweighart2002}によると,ほげほげ.
また,ふがふがはぴよぴよである\cite{Hadaegh2016}.


\renewcommand{\refname}{}
\phantomsection
\addcontentsline{toc}{chapter}{参考文献}
\section*{参考文献}
% レイアウト調整
\vspace{-20pt}
\bibliographystyle{junsrt}
\bibliography{bibdata}

\end{multicols}
\end{document}


